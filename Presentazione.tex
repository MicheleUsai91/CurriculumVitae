Sono appassionato di informatica sin da quando sono cresciuto con l'Amiga64 
e i primi anni di Windows. Ho iniziato ad assemblare PC e a fare overclock 
con Windows XP e Pentium 4. Ho imparato ad installare diversi sistemi operativi 
e a conoscere le differenze tra CMD e Bash, scoprendo la semplicità dei package 
manager rispetto all'installazione di programmi su Windows. 

Sebbene l'informatica 
fosse un hobby, mi ha permesso di acquisire competenze in altri campi lavorativi, 
come quello di Personal Trainer, dove ho sviluppato capacità di mediazione e 
coordinamento di gruppi di lavoro, utilizzando Excel per la pianificazione e il 
calcolo dei carichi di lavoro. 

Nel 2020 ho deciso di trasformare questa passione 
in un lavoro e ho seguito un corso di informatica all'Università di Cagliari, 
innamorandomi della programmazione con C. Successivamente, ho avuto l'opportunità 
di lavorare presso la società PCCube grazie al corso di Generation Italy, dove 
ho acquisito esperienza nella gestione di applicazioni semplici e complesse e ho 
imparato che la qualità di un programmatore non si limita alla scrittura di buon 
codice, ma include anche la capacità di lavorare su progetti pre-esistenti.

Nel tempo libero mi piace principalmente leggere e studiare nuovi linguaggi, 
in questo periodo, per esempio, sto sviluppando una mia applicazione, scritta 
principalmente in PHP e SQLite, per la gestione delle mie finanze, della giornata
e di altri dati. Per quanto sia in locale per ora, in seguito lo sposterò su un 
server personale, probabilmente usando Ngnix come base.